\documentclass{beamer}

\usepackage[T1]{fontenc} 
\usepackage[latin1]{inputenc}
\usepackage[frenchb]{babel}

\usepackage{graphicx}
\usepackage{wrapfig}
\usepackage{listings}

\usepackage{beamerthemesplit}
\usetheme{CambridgeUS} 

\title{LCA}
\author{Maillard Kenji}
\date{\today}

\begin{document}

\section{Le langage Rock}

\frame
{
  \titlepage
}

\frame
{
  \tableofcontents
}

\frame
{
  \frametitle{Description du langage}
  Le langage de description des circuits est un langage d�claratif bas� sur la
  notion de blocs :
  \begin{itemize}
  \item Les blocs de base : \texttt{Xor}, \texttt{And}, \texttt{Or},
    \texttt{Mux}, \texttt{Not}, \texttt{Gnd}, \texttt{Vdd} et \texttt{Reg}.
  \item Les blocs d�finis par l'utilisateur. 
  \item Les p�riph�riques ou blocs externes.
  \end{itemize}

  \vspace{1cm}
  Il est compil� � l'aide du logiciel \emph{Obsidian}.
}

\begin{frame}[fragile]
  \frametitle{Exemple d'emploi}
  \small
  \begin{lstlisting}
 HalfAdder ( a, b)
    Xor X( a, b)
    And	A( a, b)
    -> o : X.o, c : A.o ;

 ParallelAdder <1> ( a, b, c)
    HalfAdder H1( a, b)
    HalfAdder H2( c, H1.o)
    Or O( H1.c, H2.c) (a, b)
    -> o : H2.o, c : O.o ;  

 ParallelAdder <n> (a[n] , b[n], c)
    ParallelAdder<n-1> A (a[0..n-2], b[0..n-2], c)  
    FullAdder F (a[n-1], b[n-1], A.c) 
    -> o[n] : { A.o, F.o }, c : F.c ;

 start ParallelAdder < 8 >
  \end{lstlisting}
\end{frame}

\frame
{
  \frametitle{Les blocs}

}

\frame
{
  \frametitle{Les fils}

}


\frame
{
  \frametitle{Motifs \& r�cursion}

}

\frame
{
  \frametitle{Les red�finitions d'horloge}
}

\frame
{
  \frametitle{Les p�riphiques}
}

\section{Le compilateur Obsidian}

\frame
{
  \frametitle{Organisation globale}
}

\frame
{
  \frametitle{Analyse lexicale, syntaxique et s�mantique}
}

\frame
{
  \frametitle{Construction du graphe}
}

\frame
{
  \frametitle{G�n�ration du code}
}


\end{document}


