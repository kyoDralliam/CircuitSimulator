\documentclass{article}

\usepackage[T1]{fontenc} 
\usepackage[utf8]{inputenc}
\usepackage[frenchb]{babel}

\title{Rapport du Projet de Système digital \no1}
\author{Belghiti Ismaël, Desfontaines Damien, Geoffroy Guillaume, Maillard Kenji}
\date{\today}

\usepackage{amsthm}
\usepackage{amsmath}
\usepackage{amssymb}
\usepackage{mathrsfs}

\begin{document}

\maketitle

\section{Introduction}
Le projet consiste actuellement en un compilateur qui prend en argument un fichier rock (le nom du langage que nous utilisons) et crée un éxécutable qui simule le circuit.

\subsection{Plan général du compilateur}
Les différentes étapes de la compilation sont :
\begin{enumerate}
\item Lexer
\item Parseur
\item Analyse sémantique
\item Destruction de l'AST en graphe
\item Génération du code C++
\item Compilation vers un éxécutable natif
\end{enumerate}

\subsection{Prise en main rapide}
Le compilateur s'appelle \emph{obsidian}. Il attend en entrée au moins
un fichier .rock contenant la source du programme. On accéde à l'aide
via l'option \texttt{-help}. Les autres options intéressantes sont
\texttt{-pdf}, \texttt{-o}, \texttt{-cc}, \texttt{-ccflags} et
\texttt{-susucre}. Le programme résultant accepte deux options, -s
pour afficher les sorties à chaque cycle et -c n pour indiquer la
fréquence du compilateur, ainsi qu'un argument obligatoire, le nombre
de cycles. Les entrées du circuit sont attendues sur l'entrée
standard. Plusieurs exemples sont présents dans FrontEnd/tests/ et
compilable avec \texttt{make tests}.
On peut aussi simuler directement un circuit à partir de sa net-list
avec :
\begin{verbatim}
$ ./simultor.sh path-to-net-list cycles
\end{verbatim}
 



\section{Le langage rock}

Le langage de description des circuits et un langage déclaratif basé sur la notion de bloc. Il y a trois types de blocs différents :
\begin{itemize}
\item Les blocs de base : xor, and, or, mux, not, ground, vdd et reg (déclarés dans Analysis/baseBlocks.ml)
\item Les blocs classiques pouvant être paramètrés par des entiers, prenant des fils en entrées et créant des fils en sorties, qui définissent leur corps à partir d'autres blocs.
\item Les périphériques ou blocs externes.
\end{itemize}

\paragraph{Syntaxe du langage}
Le langage suit la grammaire suivante :\newline\newline
\begin{tabular}{ r c l }
  fichier &$::=$&\: instruction* BlocEntrée instruction*\\ [1.5ex]
  BlocEntrée &$::=$&\: {\bf start} NomBloc $<$ ConstanteEntière, ... ,
  ConstanteEntière  $>$\\ [1.5ex]
  instruction &$::=$&\: BlocDéfinition\\
  &&| PériphériqueDéfinition \\ [1.5ex]
  BlocDéfinition &$::=$&\: NomBloc Paramètres? Arguments? Instance* $\to$
  Sorties $;$\\ [1.5ex]
  PériphériqueDéfinition &$::=$&\: {\bf device} NomBloc Paramètres\\ [1.5ex]
  NomBloc &$::=$&\: \emph{identifiant commençant par une majuscule}\\ [1.5ex]
  Paramètres &$::=$&\: $<$ Pattern, ... , Pattern $>$\\ [1.5ex]
  Pattern &$::=$&\: $a*n+k+b$     \\ 
  &&| $b$ \\ 
  \multicolumn{3}{c}{(avec $a$, $b$ des constantes entières, $a
    \neq 1$ et $n$, $k$ des identifiants de variables)} \\ [1.5ex]
  Arguments &$::=$&\: $($ DeclarationFil, ... , DeclarationFil $)$ \\ [1.5ex]
  DeclarationFil  &$::=$&\: NomFil \\
  &&| NomFil $[$ ConstanteEntière $]$ \\ [1.5ex]
  NomFil &$::=$&\: \emph{identifiant commençant par une minuscule}\\ [1.5ex]
  Instance &$::=$&\: NomBloc NomVariableBloc $($ Fil, ... , Fil $)$ \\ [1.5ex]
  Fil &$::=$&\: NomFil\\
  &&| NomFil $[$ ConstanteEntière $]$\\
  &&| NomFil $[$ ConstanteEntière {\bf..} ConstanteEntière $]$\\
  &&| {\bf \$} $(0|1)+$
  &&| $\mathbf{\{}$ Fil, ... , Fil $\mathbf{\}}$\\ [1.5ex]
  Sorties&$::=$&\: DeclarationFil : Fil, ... , DeclarationFil : Fil
\end{tabular}


\paragraph{Opérations sur les entiers.}



\subsection{Les blocs classiques}
La syntaxe des blocs classiques est comme suit :
\begin{verbatim}
NomBloc [<param1, .. >] [(fil1, ..)]
    NomBloc NomVariable[(fil1, .. )]
    -> nomSortie1 : fil1 , .. ;
\end{verbatim}

où les crochets indiquent les parties optionnelles lorsqu'elles sont vides. \\
Les paramètres (entre chevrons) doivent être d'une des formes suivantes :
\begin{itemize}
\item$n$ où $n$ est un entier littéral
\item$ax + y + b$ où $x$ et $y$ sont des variables et $a$ et $b$ des littéraux avec $a \neq 1$
\item$x + a$ où $x$ est une variable et $a$ un littéral. 
\end{itemize}
Les paramètres peuvent notamment être employés pour faire de la récursivité.\\
Les fils en entrées et en sortie sont des fils épais dont la taille doit être indiquée entre crochets sauf si le fil est de taille $1$. Deux opérateurs permettent de travailler sur les fils épais :
\begin{itemize}
\item La construction \texttt{fil[i..j]} permet d'extraire un fil de taille $j - i + 1$ tandis que \texttt{fil[i]} extrait un fil de taille $1$.
\item La construction \texttt{\{$fil_{1}$, .. , $fil_{n}$\}} permet de construire un fil de taille la somme des tailles de fil $fil_{i}$.
\end{itemize}
On accéde à la sortie d'un bloc avec la construction \texttt{NomBloc.nomSortie}. À ce propos, les noms de blocs et de variables de blocs commencent par une majuscule tandis que les noms de fils et de variables entières commencent par une minuscule. 


\subsection{Les périphériques}

Les périphériques sont des portes dont le code est fourni au compilateur. On peut par exemple faire des périphériques ``mémoire'', ou des périphériques permettant au circuit d'intéragir avec l'utilisateur. Un périphérique prend la forme d'une classe C++, qui doit dériver de la classe \texttt{device} (dont on trove une définition dans FrontEnd/devices/device.h). Elle doit posséder les membres publics suivants :
\begin{itemize}
\item Une méthode statique \texttt{make} qui prend les mêmes arguments que le constructeur (dans la netlist, ces arguments apparaissent comme les paramètres du périphérique ; ils doivent bien sûr tous être de type \texttt{int}). Cette fonction doit instancier la classe avec \texttt{new} (en passant ses arguments au constructeur) et renvoyer le pointeur de l'instance upcasté en \texttt{device}.
\item Une méthode \texttt{cycle} de prototype :
\begin{verbatim}

virtual unsigned int cycle (unsigned int address,
                            unsigned int data,
                            char byte_enables,
                            bool write_enable,
                            bool interrupt_enable,
                            bool iack,
                            char * irq);

\end{verbatim}
Cette méthode sera appelée à chaque cycle ; ses arguments sont obtenus à partir des 67 bits d'entrée du périphérique : les 32 premiers bits vont dans \texttt{address} (en ordre \emph{little endian}), les 32 suivants dans \texttt{data}, les 4 suivants dans \texttt{byte\_enables}, et les derniers dans les 3 arguments suivants. Les 32 premiers bits de sortie du périphérique définisent la valeur de retour de la méthode, et le dernier est écrit dans \texttt{*irq}. L'idée est que les périhériques du processeur seront mappés en mémoire, et accessibles comme sur un processeur \emph{mips}, d'où la présence des quatre bits \emph{byte enable}. Un exemple de device est donné dans FrontEnd/devices/memory.cpp et utilisé par FrontEnd/tests/use-memory. Un exemple plus complexe est aussi donné dans FrontEnd/devices/bus.cpp et utilisé dans FrontEnd/tests/test-bus : ici, en plus de la mémoire, on simule des afficheurs 7 segments (mappés à partir de l'adresse 0x80000000).
\end{itemize}


\end{document}